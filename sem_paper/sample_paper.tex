%%
%% This is file `sample-sigplan.tex',
%% generated with the docstrip utility.
%%
%% The original source files were:
%%
%% samples.dtx  (with options: `sigplan')
%% 
%% IMPORTANT NOTICE:
%% 
%% For the copyright see the source file.
%% 
%% Any modified versions of this file must be renamed
%% with new filenames distinct from sample-sigplan.tex.
%% 
%% For distribution of the original source see the terms
%% for copying and modification in the file samples.dtx.
%% 
%% This generated file may be distributed as long as the
%% original source files, as listed above, are part of the
%% same distribution. (The sources need not necessarily be
%% in the same archive or directory.)
%%
%% The first command in your LaTeX source must be the \documentclass command.
\DocumentMetadata{
  lang=en,
  pdfversion=2.0,
  %pdfstandard=ua-2,
  %testphase={phase-III,firstaid,math,title}
  tagging=on,
  testphase={phase-III,firstaid,math,title}
  %tagging-setup={math/setup=mathml-SE}
}
\documentclass[sigplan,screen,nonacm]{acmart-tagged}
\usepackage{color}
\setlength {\marginparwidth }{2cm}
\usepackage[colorinlistoftodos]{todonotes}
\usepackage[
    type={CC},
    modifier={by-nc-sa},
    version={4.0},
]{doclicense}
%% NOTE that a single column version is required for 
%% submission and peer review. This can be done by changing
%% the \doucmentclass[...]{acmart} in this template to 
%% \documentclass[manuscript,screen,review]{acmart}
%% 
%% To ensure 100% compatibility, please check the white list of
%% approved LaTeX packages to be used with the Master Article Template at
%% https://www.acm.org/publications/taps/whitelist-of-latex-packages 
%% before creating your document. The white list page provides 
%% information on how to submit additional LaTeX packages for 
%% review and adoption.
%% Fonts used in the template cannot be substituted; margin 
%% adjustments are not allowed.
%%
%% \BibTeX command to typeset BibTeX logo in the docs
\AtBeginDocument{%
  \providecommand\BibTeX{{%
    \normalfont B\kern-0.5em{\scshape i\kern-0.25em b}\kern-0.8em\TeX}}}


%%
%% end of the preamble, start of the body of the document source.
\begin{document}

%%
%% The "title" command has an optional parameter,
%% allowing the author to define a "short title" to be used in page headers.
\title{Optical Character Recognition}

%%
%% The "author" command and its associated commands are used to define
%% the authors and their affiliations.
%% Of note is the shared affiliation of the first two authors, and the
%% "authornote" and "authornotemark" commands
%% used to denote shared contribution to the research.
\author{Orville "El" Anderson}
\email{and10393@umn.edu}
\affiliation{%
  \institution{Division of Science and Mathematics 
	\\
        University of Minnesota, Morris
	}
  \city{Morris}
  \state{Minnesota}
  \country{USA}
  \postcode{56267}
}

%%
%% By default, the full list of authors will be used in the page
%% headers. Often, this list is too long, and will overlap
%% other information printed in the page headers. This command allows
%% the author to define a more concise list
%% of authors' names for this purpose.
%\renewcommand{\shortauthors}{Trovato and Tobin, et al.}

%%
%% The abstract is a short summary of the work to be presented in the
%% article.
\begin{abstract}

This paper looks at Optical Character Recognition(OCR) as a means to improve usability and accessibility of Scanned Documents. The focus is on how to compare OCR implementations, the weaknesses of OCR, and how we can create tests to specifically target those weaknesses.

\end{abstract}

\doclicenseThis

%%
%% Keywords. The author(s) should pick words that accurately describe
%% the work being presented. Separate the keywords with commas.
\keywords{optical character recognition, scanned documents, visual noise, datasets}


%%
%% This command processes the author and affiliation and title
%% information and builds the first part of the formatted document.
\maketitle

\section{Introduction}
\label{sec:introduction}

Scanned documents are images of existing documents. They are typically made with a camera or scanner. We have an increasing amount of scanned documents. These documents have usability issues unique from native online documents. \todo[inline]{need to define native online documents and standardize}
Some of these issues are longer loading times, issues with searching within the document, and accessibility issues for screen readers.\footnote{one of these is not like the other...} 
One way to improve the usability of these documents is to digitize them. The most straightforward way to do this is to copy the contents of the document by hand into a new file. This does not scale well, so we look to Optical Character Recognition to automate the process. 

\section{Background}
\label{sec:background}
OCR is broken up into -- steps

The output of OCR is given in many different ways. One of the more popular outputs is as the accuracy percentage.

The output of OCR is a text document
\cite{Raj:2022,Avyodri:2022,Thorat:2022}
\todo[inline]{Formula?}

\section{Challenges}
\label{sec:body}

There are three main categories of things that make scanned documents harder to digitize. 

\subsection{Layout}
\label{sec:Layout}

Layout - too many layouts to cover them all

Documents come in many different layouts. Some papers, like this one, are organized in two columns, some have one, some have sidebars and images. There exists too many ways to format a page to make an OCR well suited for all of them.

\subsection{Alphabet}
\label{sec:Alphabet}

Most OCR are trained on the Latin Alphabet. This inherently gives all other languages a bit of a disadvantage. Other alphabets also have nuances that make it harder to directly apply OCR to. A common example is Arabic. (there's dots) \cite{Fateh:2024, Hegghamer:2022}

\subsection{Visual Noise}
\label{sec:Noise}

There’s so many ways to scan a document badly. \cite{Hegghamer:2022}

\section{Results}
\label{sec:Results}

Results - some specialized datasets exist \cite{Fateh:2024,Hegghamer:2022}

\section{Conclusion}
\label{sec:Conclusion}

Conclusion - While it’s ideal to have one golden benchmarking set, that's hard (because of the reasons outlined above), so now we have specialized ones.
Acknowledgments

%% Elena: examples like this are actually helpful, but I can't locate theit bibliography 
%% file (I think they are using a databse), so providing examples will have to wait.

%  Some examples.  A paginated journal article \cite{Abril07}, an
%  enumerated journal article \cite{Cohen07}, a reference to an entire
%  issue \cite{JCohen96}, a monograph (whole book) \cite{Kosiur01}, a
%  monograph/whole book in a series (see 2a in spec. document)
%  \cite{Harel79}, a divisible-book such as an anthology or compilation
%  \cite{Editor00} followed by the same example, however we only output
%  the series if the volume number is given \cite{Editor00a} (so
%  Editor00a's series should NOT be present since it has no vol. no.),
%  a chapter in a divisible book \cite{Spector90}, a chapter in a
%  divisible book in a series \cite{Douglass98}, a multi-volume work as
%  book \cite{Knuth97}, a couple of articles in a proceedings (of a
%  conference, symposium, workshop for example) (paginated proceedings
%  article) \cite{Andler79, Hagerup1993}, a proceedings article with
%  all possible elements \cite{Smith10}, an example of an enumerated
%  proceedings article \cite{VanGundy07}, an informally published work
%  \cite{Harel78}, a couple of preprints \cite{Bornmann2019,
%    AnzarootPBM14}, a doctoral dissertation \cite{Clarkson85}, a
%  master's thesis: \cite{anisi03}, an online document / world wide web
%  resource \cite{Thornburg01, Ablamowicz07, Poker06}, a video game
%  (Case 1) \cite{Obama08} and (Case 2) \cite{Novak03} and \cite{Lee05}
%  and (Case 3) a patent \cite{JoeScientist001}, work accepted for
%  publication \cite{rous08}, 'YYYYb'-test for prolific author
%  \cite{SaeediMEJ10} and \cite{SaeediJETC10}. Other cites might
%  contain 'duplicate' DOI and URLs (some SIAM articles)
%  \cite{Kirschmer:2010:AEI:1958016.1958018}. Boris / Barbara Beeton:
%  multi-volume works as books \cite{MR781536} and \cite{MR781537}. A
%  couple of citations with DOIs:
%  \cite{2004:ITE:1009386.1010128,Kirschmer:2010:AEI:1958016.1958018}. Online
%  citations: \cite{TUGInstmem, Thornburg01, CTANacmart}. Artifacts:
%  \cite{R} and \cite{UMassCitations}.


%%
%% The acknowledgments section is defined using the "acks" environment
%% (and NOT an unnumbered section). This ensures the proper
%% identification of the section in the article metadata, and the
%% consistent spelling of the heading.
\begin{acks}
Thanks.
\end{acks}

%%
%% The next two lines define the bibliography style to be used, and
%% the bibliography file.
\bibliographystyle{ACM-Reference-Format}
\bibliography{sample_paper}


\end{document}
\endinput
%%
%% End of file `sample-sigplan.tex'.
