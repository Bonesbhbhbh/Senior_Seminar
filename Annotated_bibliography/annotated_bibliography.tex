% This is a sample document using the University of Minnesota, Morris, Computer Science
% Senior Seminar modification of the ACM sig-alternate style to generate a simple annotated
% bibliography. The idea is that this document is fairly short, consisting of a brief description
% of your sources and how you intend to use them (or not). Most of the ``content'' of the
% generated document comes from the bibliography file, including the notes field which will
% provide the annotations.

% See https://github.com/UMM-CSci/Senior_seminar_templates for more info and to make
% suggestions and corrections.

\documentclass{sig-alternate}

\usepackage{url}

\begin{document}

% --- Author Metadata here ---
%%% REMEMBER TO CHANGE THE SEMESTER AND YEAR AS NECESSARY
\conferenceinfo{UMM CSci Senior Seminar Conference, December 2015}{Morris, MN}

\title{Something about NP-complete problems}

\numberofauthors{1}

\author{
% The command \alignauthor (no curly braces needed) should
% precede each author name, affiliation/snail-mail address and
% e-mail address. Additionally, tag each line of
% affiliation/address with \affaddr, and tag the
% e-mail address with \email.
\alignauthor
Chris S. Student\\
	\affaddr{Division of Science and Mathematics}\\
	\affaddr{University of Minnesota, Morris}\\
	\affaddr{Morris, Minnesota, USA 56267}\\
	\email{cssxxxx00000@morris.umn.edu}
}

\maketitle

\begin{abstract}
This paper discusses new results in NP-complete problems and the use of distributed
networks to solve certain partial cases of NP-complete problems.
\end{abstract}

\section{Discussion of sources}

I will focus on using the \emph{super-new-nifty-method} for solving partial cases of
NP-complete problems on distributed networks.

\subsection{Sources I expect to use (and how)}

I plan to use the following sources:
\begin{itemize}
\item I expect~\cite{OM:2008, prakash2013fractional} to be two of my main sources, and I'm still looking for one more ``core'' paper to build on. \cite{OM:2008} covers \emph{this} and \emph{that}, which is important for \emph{the other}. \cite{prakash2013fractional} takes a very different approach which appears to take better advantage of some new developments in cloud infrastructure. One area where a new paper would be helpful would be in better connecting and comparing these two techniques.
\item I think I'll use~\cite{Brun:2008} for background on cloud infrastructure. 
\item I'll use~\cite{Aaronson:2005, wiki:np-complete} and possibly selected chapters of~\cite{Garey:1979}
\end{itemize}

As mentioned above I still one more ``core'' papers, and I'm still looking for good examples that I can use to explain why solving partial cases of NP-complete problems matters.

\subsection{Sources I doubt I'll use}

I was initially considering algorithms on compete graphs as a possible topic, and looked over~\cite{winkler1984isometric, dobkin1987delaunay, folkman1970graphs} before I settled on my current topic. \cite{winkler1984isometric} was quite readable and provided a nice background on complete graphs, and might still become a background citation.

I also looked at~\cite{trulyFrightening2011} which I thought would be very helpful, but turned out to be \emph{very} poorly written to the point of being almost incomprehsible. I also thought~\cite{littlePoster2013} looked promising, but it was just a two page poster paper and had almost no useful detail. I tried searching for follow-up work and couldn't find any. Some searching with my advisor suggests that this poster was part of someone's Master's thesis, and it looks like they took an industry job right after this and haven't published anything since.

% The following two commands are all you need to
% produce the bibliography for the citations in your paper.
\bibliographystyle{abbrv}
% annotated_bibliography.bib is the name of the BibTex file containing 
% all the bibliography entries for this example. Note that you *don't* include the .bib ending
% in the \bibliography command.
\bibliography{annotated_bibliography}  

% You must have a ".bib" file and remember to run:
%     pdflatex bibtex pdflatex pdflatex
% in order to see all the citation references correctly.

\end{document}


